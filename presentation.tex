% titlepage-demo.tex
\documentclass{beamer}

% items enclosed in square brackets are optional; explanation below
\title{Flow-Insensitive Points-To Analysis with Term and Set
  Constraints \cite{Foster97flow-insensitivepoints-to}}
\subtitle{Pointer analysis in type theory's clothing!}
\author{Mihir Mehta}
\institute{
  Department of Computer Science\\
  University of Texas, Austin\\
  \texttt{mihir@cs.utexas.edu}
}
\date{20 October, 2014}

\AtBeginSection[]
{
  \begin{frame}<beamer>
    \frametitle{Outline for section \thesection}
    \tableofcontents[currentsection]
  \end{frame}
}

\begin{document}

%--- the titlepage frame -------------------------%
\begin{frame}[plain]
  \titlepage
\end{frame}

\section{Term and set constraints}

\begin{frame}{Set constraints}
  \begin{itemize}
  \item Useful for a whole lot of applications
    \cite{DBLP:journals/scp/Aiken99} ranging from register allocation
    to type inference.
  \item First formalised \cite{Heintze91adecision} in 1991, solved for good
    \cite{DBLP:conf/focs/CharatonikP94} in 1994.
    \begin{definition}
      Set constraints describe relationships between sets of
      \emph{terms}. They take the form $X \subseteq Y $, where $X$ and
      $Y$ are set expressions, generated by this grammar.
      $E ::= \alpha | 
      0 | 
      E_1 \cup E_2 | 
      E_1 \cap E_2 | 
      \neg E_1 |
      c(E_1,\dotsc, E_{a(c)})|
      c^{-i}(E_1)$
    \end{definition}
  \end{itemize}
\end{frame}

\begin{frame} Test \end{frame}

\section{Test section two}
\begin{frame} Test \end{frame}
\begin{frame} Test \end{frame}

\begin{frame}[allowframebreaks]
  \frametitle{References}
  \bibliographystyle{plain}
  \bibliography{citeseerx,Aiken99,CharatonikP94}
\end{frame}

\end{document}
